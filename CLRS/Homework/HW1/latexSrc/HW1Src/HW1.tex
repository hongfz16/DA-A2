%!TEX program = xelatex
%!TEX TS-program = xelatex
%!TEX encoding = UTF-8 Unicode

\documentclass[12pt]{article}

\usepackage{amsmath}
\usepackage{fontspec,xltxtra,xunicode}
\usepackage{geometry}
\usepackage{hyperref}
\usepackage{booktabs}

\defaultfontfeatures{Mapping=tex-text}
\setromanfont{Heiti SC}
\XeTeXlinebreaklocale “zh”

\newfontfamily{\H}{Songti SC}
\newfontfamily{\E}{Weibei SC}

\title{Homework1 Solution}
\date{\today}
\author{洪方舟\\学号: 2016013259\\Email: \href{mailto:hongfz16@163.com}{hongfz16@163.com}}

\geometry{left=2cm,right=2cm,top=2cm,bottom=2cm}

\begin{document}
  \maketitle
  \section*{Problem1}
    \subsection*{a. Prove $ 2n + \Theta(n^2) = \Theta(n^2) $ \\\textbf{证明:}}
      先证 $ 2n + \Theta(n^2) \subseteq \Theta(n^2) $ \\
      $ \forall f(n) \in 2n+\Theta(n^2) $, $ \exists c_{1}, c_{2}, n_{0} \in \Re^{+} $, $s.t.$ $ \forall n \geq n_{0} $
      \begin{equation}
        0 \leq c_{1}(2n+n^2) \leq f(n) \leq c_2(2n+n^2)
      \end{equation}
      令 $ n_1=max(n_0, 2) $, $ \exists n_1 \in \Re^+ $, $s.t.$ $ \forall n \geq n_1 $
      \begin{equation}
      	0 \leq c_1n^2 \leq f(n) \leq c_2(2n+n^2) \leq 2c_2n^2
      \end{equation}
      则 $ f(n) \in \Theta(n^2) $, 也即 $ 2n + \Theta(n^2) \subseteq \Theta(n^2) $\\\\
      再证 $ \Theta(n^2) \subseteq 2n + \Theta(n^2) $\\
      $ \forall f(n) \in \Theta(n^2), \exists c_1, c_2, n_0 \in \Re^+, s.t. \forall n \geq n_0 $
      \begin{equation}
      	0 \leq c_1n^2 \leq f(n) \leq c_2n^2
      \end{equation}
      令 $ n_1=max(n_0, 2), \exists n_1 \in \Re^+, s.t. \forall n \geq n_1 $
      \begin{equation}
        0 \leq \frac{c_1}{2}(2n+n^2) \leq c_1n^2 \leq f(n) \leq c_2n^2 \leq c_2(2n+n^2)
      \end{equation}
      则 $ f(n) \in 2n+\Theta(n^2) $, 也即 $ \Theta(n^2) \subseteq 2n+\Theta(n^2) $\\\\
      综上所述, $ 2n + \Theta(n^2) = \Theta(n^2) $\\
      证毕!
    \subsection*{b. Prove $ \Theta(g(n)) \cap o(g(n))=\emptyset $ \\\textbf{证明:}}
      $ \forall f(n) \in \Theta(g(n)), \exists c_1, c_2, n_0 \in \Re^+, s.t. \forall n \geq n_0 $
      \begin{equation}
        0 \leq c_1g(n) \leq f(n) \leq c_2g(n)
      \end{equation}
      假设 $ f(n) \in o(g(n)), \forall c_3 > 0, \exists n_1 \in \Re^+, s.t. \forall n \geq n_1 $
      \begin{equation}
        0 \leq f(n) < c_3g(n)
      \end{equation}
      取 $ c_3=c_1, n_2=max(n_0,n_1), s.t. \forall n \geq n_3 $
      \begin{equation}
        c_3g(n)=c_1g(n) \leq f(n)
      \end{equation}
      这与(6)式矛盾,故假设不成立,也即 $ \forall f(n) \in \Theta(g(n)), f(n) \notin o(g(n)) $\\
      综上所述,$ \Theta(g(n)) \cap o(g(n))=\emptyset $\\
      证毕!
    \subsection*{c. Prove $ \Theta(g(n)) \cup o(g(n)) \neq O(g(n)) $ \\\textbf{证明:}}
      取 $ g(n)=1, f(n)=\left|sin\left(\frac{\pi n}{2}\right)\right| $\\
      下面说明 $ f(n) \in O(g(n)) $ 且 $ f(n) \notin \Theta(g(n)) \cup o(g(n)) $\\
      取 $ c_1=1, n_0=1, \forall n \leq n_0 $
      \begin{equation}
        0 \leq f(n) \leq c_1g(n) = c_1 = 1
      \end{equation}
      则 $ f(n) \in O(g(n)) $\\
      因为 $ \forall n=2k, k=1,2,3..., f(n)=0 $, 所以 $ \forall c_2, n_0 \in \Re^+, \exists n>n_0, n=2k, s.t.$
      \begin{equation}
        f(n)=0<c_2g(n)=c_2
      \end{equation}
      也即 $ f(n) \notin \Theta(g(n)) $\\
      取 $ c_3=\frac{1}{2}, \forall n_1 \in \Re^+, \exists n>n_1, n=2k+1, s.t. $
      \begin{equation}
        \frac{1}{2} = c_3g(n) < f(n) = 1
      \end{equation}
      也即 $ f(n) \notin o(g(n)) $\\
      综上所述,$ f(n) \in O(g(n)) $ 且 $ f(n) \notin \Theta(g(n)) \cup o(g(n)) $\\
      则 $ \Theta(g(n)) \cup o(g(n)) \neq O(g(n)) $\\
      证毕!
    \subsection*{d. Prove $ max(f(n),g(n)) = \Theta(f(n)+g(n)) $ \\\textbf{证明:}}
      令 $ h(n)=max(f(n),g(n)), \exists n_0 \in \Re^+, s.t. \forall n>n_0 $
      \begin{align}
      	h(n) &\geq f(n)\\
        h(n) &\geq g(n)\\
        h(n) &\leq f(n)+f(n)+2g(n)
      \end{align}
      整理可得
      \begin{equation}
        0 \leq \frac{1}{2}(f(n)+g(n)) \leq h(n) = max(f(n),g(n)) \leq 2(f(n)+g(n))
      \end{equation}
      则可得 $ max(f(n),g(n)) = \Theta(f(n)+g(n)) $\\
      证毕!
  \section*{Problem2}
    \subsection*{a. $ T(n)=2T(\sqrt{n})+1 $ \\\textbf{解:}}
      令 $ m=lgn $, 则有 $ n=2^m $, 带入原式得
      \begin{equation}
        T(2^m)=2T(2^{\frac{m}{2}})+1
      \end{equation}
      令 $ G(n)=T(2^n) $, 可得
      \begin{equation}
        G(m)=2G(\frac{m}{2})+1
      \end{equation}
      根据主定理, $ a=2, b=2, f(n)=1 $, 则 $ n^{log_ba}=n $, 取 $ \epsilon=1 $, 则有
      \begin{equation}
        1=f(n)=\Theta(n^{log_ba-\epsilon})=\Theta(1)
      \end{equation}
      则 $ G(n)=\Theta(n) $, 又因为 $ G(n)=T(2^n) $, 可得
      \begin{equation}
        T(n)=\Theta(lgn)
      \end{equation}
    \subsection*{b. $ nT(n)=(n-2)T(n-1)+2 $ \\\textbf{解:}}
      原式可化为
      \begin{equation}
        n(T(n)-1)=(n-2)(T(n-1)-1)
      \end{equation}
      令 $ G(n)=T(n)-1 $, 则上式可化为
      \begin{equation}
        \frac{G(n)}{G(n-1)}=\frac{n-2}{n}
      \end{equation}
      累乘可得
      \begin{equation}
        \frac{G(n)}{G(2)}=\frac{G(n)}{G(n-1)}\frac{G(n-1)}{G(n-2)}...\frac{G(3)}{G(2)}=\frac{n-2}{n}\frac{n-3}{n-1}...\frac{1}{3}=\frac{2(n-2)!}{n!}=\frac{2}{n(n-1)}
      \end{equation}
      则可得
      \begin{equation}
      	T(n)=\frac{2G(2)}{n(n-1)}+1
      \end{equation}
      也即
      \begin{equation}
        T(n)=\Theta\left(\frac{1}{n^2}\right)
      \end{equation}
  \section*{Problem3}
  \subsection*{a. \textbf{解:}}
    下表给出排序结果,从上到下渐进增长率增大,同一行中从左向右渐进增长率增大
    \begin{table}[h!]
      \begin{center}
        \begin{tabular}{l|c}
          \toprule
          \textbf{等价类} & \textbf{函数}\\
          \midrule
          $ \Theta(1) $ & $ 1, n^{\frac{1}{lgn}} $\\
          $ \Theta(lg(lg^*n)) $ & $ lg(lg^*n) $\\
          $ \Theta(lg^*(lgn)) $ & $ lg^*(lgn) $\\
          $ \Theta(lg^*n) $ & $ lg^*n $\\
          $ \Theta(2^{lg^*n}) $ & $ 2^{lg^*n} $\\
          $ \Theta(lg(lgn)) $ & $ ln(lnn) $\\
          $ \Theta(\sqrt{lgn}) $ & $ \sqrt{lgn} $\\
          $ \Theta(lgn) $ & $ lnn $\\
          $ \Theta(lg^2n) $ & $ lg^2n $\\
          $ \Theta(2^{\sqrt{2lgn}}) $ & $ 2^{\sqrt{2lgn}} $\\
          $ \Theta(n^{\frac{1}{2}}) $ & $ \sqrt{2}^{lgn} $\\
          $ \Theta(n) $ & $ 2^{lgn}, n $\\
          $ \Theta(nlgn) $ & $ nlgn, lg(n!) $\\
          $ \Theta(n^2) $ & $ n^2, 4^{lgn} $\\
          $ \Theta(n^3) $ & $ n^3 $\\
          $ \Theta((lgn)!) $ & $ (lgn)! $\\
          $ \Theta(lgn^{lgn}) $ & $ lgn^{lgn}, n^{lg(lgn)} $\\
          $ \Theta(\left(\frac{3}{2}\right)^{n}) $ & $ \left(\frac{3}{2}\right)^{n} $\\
          $ \Theta(2^n) $ & $ 2^n $\\
          $ \Theta(n2^n) $ & $ n2^n $\\
          $ \Theta(e^n) $ & $ e^n $\\
          $ \Theta(n!) $ & $ n! $\\
          $ \Theta((n+1)!) $ & $ (n+1)! $\\
          $ \Theta(2^{2^{n}}) $ & $ 2^{2^{n}} $\\
          $ \Theta(2^{2^{n+1}}) $ & $ 2^{2^{n+1}} $\\
          \bottomrule
        \end{tabular}
      \end{center}
    \end{table}
  \subsection*{b. \textbf{解:}}
    令 $ f(n)=2^{2^{n+2}}+(-1)^n2^{2^{n+2}} $, 当$n$为偶数时, $ f(n)=\Theta(2^{2^{n+2}}) $, 当$n$为奇数时, $ f(n)=0 $\\
    因此, $ \forall c_1, c_2, n_0 \in \Re^+, \exists n>n_0, s.t. $
    \begin{align}
      f(n) &> c_1g_i(n) \quad (n=2k)\\
      f(n) &< c_2g_i(n) \quad (n=2k+1)
    \end{align}
    则对于所有$g_i(n)$, 上述$f(n)$既不是$ O(g_i(n)) $也不是$ \Omega(g_i(n)) $
\end{document}