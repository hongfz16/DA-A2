%!TEX program = xelatex
%!TEX TS-program = xelatex
%!TEX encoding = UTF-8 Unicode
\documentclass[12pt]{article}

\usepackage{amsmath}
\usepackage{hyperref}
\usepackage{geometry}
\usepackage{fontspec,xltxtra,xunicode}
\usepackage{booktabs}
\usepackage{float}
\usepackage{changepage}

\defaultfontfeatures{Mapping=tex-text}
\setromanfont{Heiti SC}
\XeTeXlinebreaklocale “zh”

\title{矩阵乘法比较实验报告}
\author{洪方舟\\Student ID: 2016013259\\Email: \href{mailto:hongfz16@163.com}{hongfz16@163.com}}

\geometry{left=2cm,right=2cm,top=2cm,bottom=3cm}

\begin{document}
  \maketitle
  \section*{1. 实验目的}
  a. 实现两种矩阵相乘的方法(常规法与Strassen方法)\\
  b. 比较两种矩阵相乘方法在不同数据大小(矩阵维数)下所需时间\\
  c. 分析实验结果并给出在实际操作中选择哪种方法建议
  \section*{2. 实验环境}
  操作系统:macOS High Sierra (Version 10.13.1)\\
  处理器:1.6GHz Intel Core i5\\
  编程语言:C++\\
  编译器:g++
  \section*{3. 实验方法}
  a. 编写两种不同的计算矩阵的函数,将两种方法的计算结果进行比较来保证正确率\\
  b. 随机生成两个维数为$2^n, n=1,2,3,...,11$的矩阵,使用上述两种方法进行计算,并计算时间\\
  c. 记录不同维数下两种方法所耗时间,进行比较分析
  \newpage
  \section*{4. 实验结果}
    \begin{table}[h!]
      \begin{center}
        \caption{Strassen方法实现中,递推的结束点设置为矩阵维数为$1\times1$}
        \begin{tabular}{c|c|c}
          \toprule
          \textbf{矩阵维数} & \textbf{常规法耗时($s$)} & \textbf{Strassen方法耗时($s$)}\\
          \midrule
          $2$ & $7e-06$ & $3.5e-05$\\
          $4$ & $2e-06$ & $0.000178$\\
          $8$ & $6e-06$ & $0.001375$\\
          $16$ & $3e-05$ & $0.008691$\\
          $32$ & $0.00023$ & $0.058942$\\
          $64$ & $0.001829$ & $0.281699$\\
          $128$ & $0.010066$ & $2.17835$\\
          $256$ & $0.149735$ & $15.0091$\\
          $512$ & $1.35245$ & $105.976$\\
          $1024$ & $33.5958$ & $717.901$\\
          $2048$ & $307.071$ & $\times$\\
          \bottomrule
        \end{tabular}
      \end{center}
    \end{table}
    显然,虽然理论上Strassen方法的时间复杂度$\Theta(n^{2.81})$低于常规方法的$O(n^3)$,但是实际运行结果表明后者效率远高于前者。\\
    \begin{table}[H]
      \begin{center}
        \caption{Strassen方法实现中,递推的结束点设置为矩阵维数为$128\times128$}
        \begin{tabular}{c|c|c}
          \toprule
          \textbf{矩阵维数} & \textbf{常规法耗时($s$)} & \textbf{Strassen方法耗时($s$)}\\
          \midrule
          $2$ & $3.7e-05$ & $2e-06$\\
          $4$ & $3e-06$ & $3e-06$\\
          $8$ & $8e-06$ & $9e-06$\\
          $16$ & $4.6e-05$ & $4.6e-05$\\
          $32$ & $0.00035$ & $0.000352$\\
          $64$ & $0.002738$ & $0.002013$\\
          $128$ & $0.016678$ & $0.016214$\\
          $256$ & $0.100411$ & $0.078303$\\
          $512$ & $1.05642$ & $0.57834$\\
          $1024$ & $10.6463$ & $4.3006$\\
          $2048$ & $213.335$ & $29.9343$\\
          $4096$ & $2218.87$ & $216.887$\\
          \bottomrule
        \end{tabular}
      \end{center}
    \end{table}
    对递归结束点进行调整之后,Strassen方法的运行效率明显提高,理论上的复杂度优势得到了明显的体现。
  \section*{5. 分析与总结}
  a. 上述第一个实验中Strassen方法明显慢于常规方法,可能的原因包括:
  \begin{adjustwidth}{1cm}{1cm}
  $i.$Strassen方法在计算中需要动态分配大量的内存,每递归一层就需要分配$\Theta(n^2)$的内存,大量的时间耗费在分配内存空间上\\
  $ii.$Strassen方法使用递归的思想,多次的递归调用耗时较多
  \end{adjustwidth}
  b. 第二个实验中将Strassen方法递归的结束点设置为$128 \times 128$之后,减少了递归调用层数,从而大量减少了所需的分配空间的时间,这样做可以减小复杂度中的常数,从而在与常规方法的比较中体现出复杂度的优势,在较大维数下可以达到常规方法十倍的速度。\\
  c. 综上,在实际运用中,对于维数小于256的矩阵应该使用常规方法,而当维数高于256的时候,应当使用Strassen方法,且须将递归的结束条件设置为维数为$128\times128$
  \section*{6. 源代码及可执行文件说明}
  源代码存放在$/bin$目录下,可执行文件存放在$/src$目录下,由于在$macOS$环境下进行编译,所以可执行文件只能在$macOS$环境中运行。运行后自动开始测试不同维数的矩阵相乘,在控制台中输出当前维数信息并且输出两种方法所用时间。
\end{document}