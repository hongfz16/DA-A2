%!TEX program = xelatex
%!TEX TS-program = xelatex
%!TEX encoding = UTF-8 Unicode
\documentclass[12pt]{article}

\usepackage{amsmath}
\usepackage{hyperref}
\usepackage{geometry}
\usepackage{fontspec,xltxtra,xunicode}
\usepackage{booktabs}
\usepackage{float}
\usepackage{changepage}
\usepackage{setspace}

\defaultfontfeatures{Mapping=tex-text}
\setromanfont{Heiti SC}
\XeTeXlinebreaklocale “zh”

\title{矩阵乘法比较实验报告}
\author{洪方舟\\Student ID: 2016013259\\Email: \href{mailto:hongfz16@163.com}{hongfz16@163.com}}

\geometry{left=2cm,right=2cm,top=2cm,bottom=3cm}

\renewcommand{\baselinestretch}{1.5}

\begin{document}
  \maketitle
  \section*{1. 实验目的}
  a. 编写程序实现查找点集中最近点对\\
  b. 对比$O(n^2)$算法与$O(nlgn)$算法的实际运行时间\\
  c. 通过对递归深度的限制优化实现$O(nlgn)$算法的运行效率最大化
  \section*{2. 实验环境}
  操作系统:macOS High Sierra (Version 10.13.1)\\
  处理器:1.6GHz Intel Core i5\\
  编程语言:C++\\
  编译器:g++
  \section*{3. 实验方法}
  a. 随机生成不同数量级的点集,分别使用两种算法运算最近点对,并测出运行时间\\
  b. 通过将两种方法的结果对比来确保算法的正确性
  \newpage
  \section*{4. 实验结果}
    \begin{table}[h!]
      \begin{center}
        \caption{分治方法中,设置递归在点集大小为$1$时停止}
        \begin{tabular}{c|c|c}
          \toprule
          \textbf{点集大小} & \textbf{分治法耗时($ms$)} & \textbf{常规方法耗时($ms$)}\\
          \midrule
          $10$ & $0.014$ & $0$\\
		  $100$ & $0.134$ & $0.025$\\
		  $1000$ & $1.466$ & $2.331$\\
		  $10000$ & $18.698$ & $236.863$\\
		  $100000$ & $191.908$ & $19319.7$\\
		  $1000000$ & $1796.85$ & $X$\\
          \bottomrule
        \end{tabular}
      \end{center}
    \end{table}
    \begin{table}[h!]
      \begin{center}
        \caption{分治方法中,设置递归在点集大小为$100$时停止}
        \begin{tabular}{c|c|c}
          \toprule
          \textbf{点集大小} & \textbf{分治法耗时($ms$)} & \textbf{常规方法耗时($ms$)}\\
          \midrule
          $10$ & $0.004$ & $0.001$\\
		  $100$ & $0.038$ & $0.023$\\
		  $1000$ & $0.532$ & $2.288$\\
		  $10000$ & $6.554$ & $159.721$\\
		  $100000$ & $69.454$ & $19588.7$\\
		  $1000000$ & $769.546$ & $X$\\
          \bottomrule
        \end{tabular}
      \end{center}
    \end{table}
  \section*{5. 分析与总结}
  a. 在“最近点对”的实验中可以看到,分治方法$O(nlgn)$复杂度相对于常规方法是一个很大的提升\\
  b. 由于常数并不是非常的大,所以递归深度的影响不是很严重\\
  c. 通过反复试验,将递归设置在点集大小为$100$时停止,得到的效率是最高的,优化后时间能够减少一半\\
  \section*{6. 源代码及可执行文件说明}
  \subsection*{a. 控制台程序}
    1. 在$/src/TClosestPair$目录下存放着控制台程序的源代码,使用$makefile$来组织项目,在命令行中使用$make$命令,将会在同目录下产生名为$main$的可执行文件;\\
    2. 在$/bin/termimal$目录下分别存放着可以在$Windows$和$macOS$环境下运行的可执行文件。
  \subsection*{b. GUI}
  	1. 可以在$Windows$环境下运行的$GUI$程序存放在$/bin/GUI$目录下;\\
  	2. 该程序支持两种输入点的方式,一种是输入数字,随机自动生成相应数量的点;另一种是点击屏幕输入点,若使用该方式输入点,请先点击$Start Click Input$按钮,在屏幕上点选过后,再点击$End Click Input$按钮;\\
  	3. 程序提供两种求最近点对的方法,对应按钮为$O(nlogn) Get Closest Pair$和$O(n^2) Get Closest Pair$,输入点对完成之后可以点击两个按钮之中的一个来计算最近点对,注意如果是点击屏幕输入点对的方式,请一定在输入完成之后点击$End Click Input$再继续下面的计算操作;\\
  	4. $Cancel$按钮负责清除计算结果,$Clear Points$将会把屏幕上的点清空;\\
  	5. 计算点对完成之后,将会在屏幕中连接两点,标出那两点的坐标,并且在屏幕左上方给出最近距离;\\
    6. 源代码保存于$/src/QtClosestPair$目录中;
\end{document}