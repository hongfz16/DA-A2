%!TEX program = xelatex
%!TEX TS-program = xelatex
%!TEX encoding = UTF-8 Unicode
\documentclass[12pt]{article}

\usepackage{amsmath}
\usepackage{hyperref}
\usepackage{geometry}
\usepackage{fontspec,xltxtra,xunicode}
\usepackage{setspace}

\defaultfontfeatures{Mapping=tex-text}
\setromanfont{Heiti SC}
\XeTeXlinebreaklocale “zh”

\title{Homework2 Solution}
\author{洪方舟\\Student ID: 2016013259\\Email: \href{mailto:hongfz16@163.com}{hongfz16@163.com}}

\geometry{left=2cm,right=2cm,top=1cm,bottom=2cm}
\renewcommand{\baselinestretch}{1.5}

\begin{document}
  \maketitle
  \section*{2. \textbf{证明:}}
  	设$E_{ij}(i<j)$为数组$P$中$P[i]$与$P[j]$元素相同的事件,存在元素不唯一的概率可以表达为$P\left(\bigcup\limits_{i<j} E_{ij}\right)$,则所有元素都唯一的概率$P(x)$为
  	\begin{equation*}
  	  P(x)=1-P\left(\bigcup\limits_{i<j} E_{ij}\right) \geq 1-\sum_{i<j} P(E_{ij})
  	\end{equation*}
  	又因为$P(E_{ij})=\frac{1}{n^3}$,所以
  	\begin{align*}
  	  P(x) &\geq 1 - \frac{n(n-1)}{2}\frac{1}{n^3}\\
  	  &=1-\frac{1}{2n}+\frac{1}{2n^2}\\
  	  &\geq 1 - \frac{1}{n}
  	\end{align*}
  	则数组$P$中所有元素都唯一的概率至少是$1-\frac{1}{n}$,证毕!
  \section*{3. \textbf{解:}}
    \subsection*{a.}
      取出恰好为$i$的概率为$\frac{1}{n-2}$,从剩下的$n-1$个数中要一个数在前$i-1$个,另一个在后$n-i$个数中,则可得$p_i$的准确表达式为
      \begin{equation*}
        p_i=\frac{6(i-1)(n-i)}{(n-2)(n-1)n}
      \end{equation*}
    \subsection*{b.}
      若选择$x=A'[\lfloor (n+1)/2 \rfloor]$,则有
      \begin{align*}
        p_x &= \frac{6\left( \lfloor \frac{n+1}{2} \rfloor -1 \right)(n-\lfloor \frac{n+1}{2} \rfloor)}{(n-2)(n-1)n}\\
        &=\left\{
        \begin{aligned}
          \frac{3(n-1)}{2n(n-2)},当n为奇数时\\
          \frac{3}{2(n-1)},当n为偶数时
        \end{aligned}
        \right.
      \end{align*}
      平凡实现中,选择$x=A'[\lfloor (n+1)/2 \rfloor]$的概率为$p_{平凡}=\frac{1}{n}$,将两者相除可得三数取中法概率增加为
      \begin{align*}
        \frac{p_x}{p_{平凡}}=\left\{
        \begin{aligned}
          \frac{3(n-1)}{2(n-2)},当n为奇数\\
          \frac{3n}{2(n-1)},当n为偶数
        \end{aligned}
        \right.
      \end{align*}
      假设$n \rightarrow \infty$,$p_x=0$,$\frac{p_x}{p_{平凡}}=\frac{3}{2}$
    \subsection*{c.}
      \begin{align*}
        Pr\{x=A'[i],n/3 \leq i \leq 2n/3\} &= \sum_{i=n/3}^{2n/3}p_i\\
        &=\sum_{i=n/3}^{2n/3}\frac{6(i-1)(n-i)}{(n-2)(n-1)n}\\
        &=\int_{n/3}^{2n/3}\frac{6(x-1)(n-x)}{(n-2)(n-1)n}dx\\
        &=\frac{13n^2-27n}{27(n-1)(n-2)}
      \end{align*}
    平凡实现中,$p_{平凡}=\frac{1}{3}$,显然有$Pr\{x=A'[i],n/3 \leq i \leq 2n/3\} > \frac{1}{3}$,增加的概率为
      \begin{align*}
        \frac{Pr\{x=A'[i],n/3 \leq i \leq 2n/3\}}{p_{平凡}}=\frac{13n^2-27n}{9(n-1)(n-2)}
      \end{align*}
    \subsection*{d.}
      对于在证明快排期望时间的时候所定义的指示变量$X_k$,平凡方法中$E[X_k]=1/n$,而在三数取中法中为$E[X_k]=\frac{6(k-1)(n-k)}{(n-2)(n-1)n}=\Theta\left(\frac{1}{n}\right)$,所以三数取中法并没有对证明过程产生数量级上的影响,因此三数取中法所得的最后结果仍然为$\Omega(nlgn)$,因此三数取中法仅影响常数项因子。
\end{document}