%!TEX program = xelatex
%!TEX TS-program = xelatex
%!TEX encoding = UTF-8 Unicode

%\documentclass[12pt]{article}
%\usepackage{amsmath}
%\usepackage{hyperref}
%\usepackage{geometry}
%\usepackage{fontspec,xltxtra,xunicode}
%\usepackage{booktabs}
%\usepackage{float}
%\usepackage{changepage}
%\usepackage{setspace}
%\usepackage{listings}

\documentclass{article}
\usepackage{amsmath}
\usepackage{geometry}
\usepackage{CJK}
\usepackage{ctex}
\usepackage{graphicx}
\usepackage{booktabs}
\usepackage{pst-node}
\usepackage{pst-text}
\usepackage{amsmath}
\usepackage{amssymb}
\usepackage{listings}
\usepackage{fancyhdr}
\usepackage[colorlinks]{hyperref}
\usepackage{color}
\usepackage{hyperref}

%\defaultfontfeatures{Mapping=tex-text}
%\setromanfont{Heiti SC}
%\XeTeXlinebreaklocale “zh”

% \definecolor{gray}{rgb}{0.4,0.4,0.4}
% \definecolor{darkblue}{rgb}{0.0,0.0,0.6}
% \definecolor{cyan}{rgb}{0.0,0.6,0.6}
% \definecolor{grey}{rgb}{0.7,0.7,0.7}

\title{Homework4 Solution}
\author{洪方舟\\2016013259\\Email: \href{mailto:hongfz16@163.com}{hongfz16@163.com}}
\date{\today}
\geometry{left=2cm,right=2cm,top=2cm,bottom=2cm}

\begin{document}
  \maketitle
  \setlength\parindent{0em}
  \section*{Problem 16-2}
    \subsection*{a.}
        \subsubsection*{算法设计:}
            将所有任务按照执行时间从小到大排序,就按照这个顺序执行,得到的平均运行结束时间是最小的
        \subsubsection*{正确性证明:}
            首先,观察到该问题有最优子结构,如果我们选择第一个执行的任务的时候就选择最优解,在选择第二个以及之后的执行任务顺序时,按照同样的方式选择执行任务的顺序,那么将会得到最优解。下面说明如果每次选择任务的时候贪心选择耗时最短的那一个,将会达到最优。设$a$是当前待选择任务中耗时最短的那一个,$b$为剩下任务中任意一个,设解法$A$将$a$排在第一个,而解法$B$ 将$a$ 和$b$ 的执行顺序对调,对于$A$中$b$之后执行的任务运行完成时间将没有变化,但是对于$a$和$b$两个任务之间的所有任务$S$,由于在$B$ 中首先需要运行时间较长的$b$,所以$S$中所有任务的平均完成时间在解法$B$中将会长于解法$A$,因此在每次选择下一个要执行的任务时,最优的方案就是贪心的选择耗时最短的任务。因此该算法具有正确性。
        \subsubsection*{时间复杂度分析:}
            只需要对所有任务按照执行时间从小到大排序即可,因此时间复杂度为$O(nlgn)$
    \subsection*{b.}
        \subsubsection*{算法设计:}
            使用最短剩余时间调度算法,伪代码如下
            \begin{lstlisting}{xml}
            function SRTF(S)
                let tc=1, pc=MAX_NUM ,idc=-1
                let result[] be an empty array
                while not S.empty()
                    for i in S.size()
                        if tc >= S[i].r and pc > S[i].p
                            if pc != MAX_NUM and pc != 0
                                for j in range(0,S.size()-1)
                                    if S[i].p <= pc and pc <= S[i+1].p
                                        S.insert(s(r=0,p=pc),i)
                                        break
                            pc=S[i].p
                            idc=i
                            break
                    tc+=1
                    pc-=1
                    result.append(idc)
                return result
            \end{lstlisting}
        \subsubsection*{正确性证明:}
            不难发现,该算法实际的完成顺序为$(S.r+S.p)$的升序,如果重新构造一组任务$S_n$,每个任务的执行时间对应为$(S[i].r+S[i].p)$,并且使用上一问的无抢占的执行规则,则该算法产生的平均执行时间等于$S_n$采用上一问的最优解计算得到的平均执行时间,则利用上一问的结论,可知在本小题有抢占的执行规则下,最短剩余时间调度算法能够实现平均执行时间最优。
        \subsubsection*{时间复杂度分析:}
            如果令$T$为完成所有任务所需的时间总长度,则外层$while$需要循环$T$次,内层$for$循环耗时$O(n)$,因此总的时间复杂度为$O(nT)$
  \section*{Problem 16-5}
    \subsection*{a.}
        使用$Farthest in Future$算法,伪代码如下
        \begin{lstlisting}{xml}
        function FFSchedule(R,k)
            let cache[k] be hashtable initalized with random ri
            let schedule[n] be new array
            for i in range(0,n)
                if R[i] in cache
                    schedule[i]=READ_FROM_CACHE
                else
                    for j=n to i+1
                        if R[j] in cache
                            schedule[i]=EVICT(R[j])
            return schedule
        \end{lstlisting}
        哈希表操作时间为$O(1)$,一共两重循环各$O(n)$,因此总的时间复杂度为$O(n^2)$
    \subsection*{b.}
        定义$R$为请求序列,$S_{ij}$为对请求序列中从$i$到$j$的请求进行的操作序列,$T_{ij}$表示$S_{ij}$中$Cache Miss$的次数。考虑对$R_{in}$的规划,若$S_{in}$为最优的调度算法:若此时$R[i]$在$Cache$中,那么此时$T_{in}=T_{(i+1)n}$,也即此时$S_{in}$中必然包含$S_{(i+1)n}$的最优子问题;若此时$R[i]$不在$Cache$中,那么就产生一次$Cache Miss$,则此时$T_{in}=1+T_{(i+1)n}$,也即此时$S_{in}$包含$S_{(i+1)n}$的最优子问题。综上,该问题具有最优子结构。
    \subsection*{c.}
        假设$S_{FF}$为按照$Farthest in Future$规则规划的序列,$S^*$为具有最少$Cache Miss$的序列;下面证明,可以通过不增加$Cache Miss$数的一个过程将$S^*$转化为$S_{FF}$;为了证明这个命题,下面证明一个该命题的递推版本:假设$S$为与$S_{FF}$前$j$个操作相同的序列,那么存在$S'$使得它与$S_{FF}$的前$j+1$个操作相同,但是$Cache Miss$数不多于$S$。\\
        设$d=d_{j+1}$,如果$d$存在于$S$和$S_{FF}$的$Cache$中,或者$d$不存在与两者的$Cache$中,但是$S$和$S_{FF}$在第$j+1$步弹出了相同的元素,那么此时$S,S',S_{FF}$在前$j+1$个操作中均相同。\\
        如果$d$不在$S$的$Cache$中,并且$S$弹出$f$,$S_{FF}$弹出$e \neq f$,下面就要找出一个$S'$,使得在第$k>j$次操作之后拥有和$S$第$k$次操作之后相同的$Cache$,在此之后只需要采取和$S$一致的操作就可以实现$k$次操作之后的操作中$Cache Miss$数和$S$相同,下面只需要说明在$j$到$k$之间做的一系列构造,使得$S'$在第$j+1$次操作和$S_{FF}$一样,并且$Cache Miss$数不多于$S$。\\
        考虑$d'=d_{j+2}$;若$d'\neq e,f$,且$S$此时弹出$e$,那么此时让$S'$弹出$f$,则此时$S'$和$S$有相同的$Cache$;若$S$弹出$h \neq e$,那么让$S'$也弹出$h$,重复本步骤继续往下找;若$d'=f$,且$S$弹出$e$,那么此时$S'$无需做任何操作已经达到相同$Cache$;若$S$弹出$e' \neq e$,那么让$S'$同样也弹出$e'$,那么也可以达到相同$Cache$。\\
        综上,已经证明了“假设$S$为与$S_{FF}$前$j$个操作相同的序列,那么存在$S'$使得它与$S_{FF}$的前$j+1$个操作相同,但是$Cache Miss$数不多于$S$”的命题,那么只需要递归的构造下去,总可以将$S^*$转化为$S_{FF}$,并且不增加$Cache Miss$,那么就可以说明$Farthest in Future$方法是最优的。
\end{document}
