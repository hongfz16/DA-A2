\documentclass{article}
\usepackage{amsmath}
\usepackage{geometry}
\usepackage{CJK}
\usepackage{ctex}
\usepackage{graphicx}
\usepackage{booktabs}
\usepackage{pst-node}
\usepackage{pst-text}
\usepackage{amsmath}
\usepackage{amssymb}
\usepackage{listings}
\usepackage{fancyhdr}
\usepackage[colorlinks]{hyperref}
\usepackage{color}
\usepackage{hyperref}

\definecolor{gray}{rgb}{0.4,0.4,0.4}
\definecolor{darkblue}{rgb}{0.0,0.0,0.6}
\definecolor{cyan}{rgb}{0.0,0.6,0.6}
\definecolor{grey}{rgb}{0.7,0.7,0.7}

\title{Homework4 Solution}
\author{洪方舟\\2016013259\\Email: \href{mailto:hongfz16@163.com}{hongfz16@163.com}}
\date{\today}
\geometry{left=2cm,right=2cm,top=2cm,bottom=2cm}
\begin{document}
  \maketitle
  \section*{Problem 16-2}
    \subsection*{a.}
        \subsubsection*{算法设计:}
            将所有任务按照执行时间从小到大排序,就按照这个顺序执行,得到的平均运行结束时间是最小的
        \subsubsection*{正确性证明:}
            首先,观察到该问题有最优子结构,如果我们选择第一个执行的任务的时候就选择最优解,在选择第二个以及之后的执行任务顺序时,按照同样的方式选择执行任务的顺序,那么将会得到最优解。下面说明如果每次选择任务的时候贪心选择耗时最短的那一个,将会达到最优。设$a$是当前待选择任务中耗时最短的那一个,$b$为剩下任务中任意一个,设解法$A$将$a$排在第一个,而解法$B$ 将$a$ 和$b$ 的执行顺序对调,对于$A$中$b$之后执行的任务运行完成时间将没有变化,但是对于$a$和$b$两个任务之间的所有任务$S$,由于在$B$ 中首先需要运行时间较长的$b$,所以$S$中所有任务的平均完成时间在解法$B$中将会长于解法$A$,因此在每次选择下一个要执行的任务时,最优的方案就是贪心的选择耗时最短的任务。因此该算法具有正确性。
        \subsubsection*{时间复杂度分析:}
            只需要对所有任务按照执行时间从小到大排序即可,因此时间复杂度为$O(nlgn)$
    \subsection*{b.}
        \subsubsection*{算法设计:}
            使用最短剩余时间调度算法,伪代码如下
            \begin{lstlisting}{xml}
            function SRTF(S)
                let tc=1, pc=\infty ,idc=-1
                let result[] be an empty array
                while not S.empty()
                    for i in S.size()
                        if tc \geq S[i].r and pc > S[i].p
                            if pc \neq \infty and pc \neq 0
                                for j in range(0,S.size()-1)
                                    if S[i].p \leq pc and pc \leq S[i+1].p
                                        S.insert(s(r=0,p=pc),i)
                            pc=S[i].p
                            idc=i
                            break
                    tc+=1
                    pc-=1
                    result.append(idc)
                return result
            \end{lstlisting}
        \subsubsection*{正确性证明:}
            不难发现,该算法实际的完成顺序为$(S.r+S.p)$的升序,如果重新构造一组任务$S_n$,每个任务的执行时间对应为$(S[i].r+S[i].p)$,并且使用上一问的无抢占的执行规则,则该算法产生的平均执行时间等于$S_n$采用上一问的最优解计算得到的平均执行时间,则利用上一问的结论,可知在本小题有抢占的执行规则下,最短剩余时间调度算法能够实现平均执行时间最优。
        \subsubsection*{时间复杂度分析:}
            如果令$T$为完成所有任务所需的时间总长度,则外层$while$需要循环$T$次,内层$for$循环耗时$O(n)$,因此总的时间复杂度为$O(nT)$
  \section*{Problem 16-5}
    \subsection*{a.}
    \subsection*{b.}
    \subsection*{c.}
\end{document}
