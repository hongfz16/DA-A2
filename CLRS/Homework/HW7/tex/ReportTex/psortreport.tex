%!TEX program = xelatex
%!TEX TS-program = xelatex
%!TEX encoding = UTF-8 Unicode

%\documentclass[12pt]{article}
%\usepackage{amsmath}
%\usepackage{hyperref}
%\usepackage{geometry}
%\usepackage{fontspec,xltxtra,xunicode}
%\usepackage{booktabs}
%\usepackage{float}
%\usepackage{changepage}
%\usepackage{setspace}
%\usepackage{listings}

\documentclass[12pt]{article}
\usepackage{amsmath}
\usepackage{geometry}
\usepackage{CJK}
\usepackage{ctex}
\usepackage{graphicx}
\usepackage{booktabs}
\usepackage{pst-node}
\usepackage{pst-text}
\usepackage{amsmath}
\usepackage{amssymb}
\usepackage{listings}
\usepackage{fancyhdr}
\usepackage[colorlinks]{hyperref}
\usepackage{color}
\usepackage{hyperref}
\usepackage{clrscode}
\usepackage{amsthm}

%\defaultfontfeatures{Mapping=tex-text}
%\setromanfont{Heiti SC}
%\XeTeXlinebreaklocale “zh”

% \definecolor{gray}{rgb}{0.4,0.4,0.4}
% \definecolor{darkblue}{rgb}{0.0,0.0,0.6}
% \definecolor{cyan}{rgb}{0.0,0.6,0.6}
% \definecolor{grey}{rgb}{0.7,0.7,0.7}

\title{Homework7}
\author{洪方舟\\2016013259\\Email: \href{mailto:hongfz16@163.com}{hongfz16@163.com}}
\date{\today}
\geometry{left=2cm,right=2cm,top=2cm,bottom=2cm}

\begin{document}
  \maketitle
  \setlength\parindent{0em}
  \section*{1. 实验目的}
  a. 使用$openmp$平台编写归并排序与快速排序的并行版本\\
  b. 通过实验验证理论是否与实际相符\\
  c. 掌握多线程编程方法
  \section*{2. 实验环境}
  操作系统:Windows 10\\
  处理器:Intel Core i7-7700k CPU @ 4.20GHz $\times$ 8\\
  编程语言:C++\\
  IDE: Visual Studio
  \section*{3. 实验方法}
  a. 首先编写无符号整形,并行版本的归并排序与快速排序\\
  b. 容易通过遍历来验证正确性\\
  c. 使用不同的数量级的数据测试两种算法\\
  d. 将测试结果与理论情况比较并进行分析\\
  e. 将测试结果与非并行版本进行比较
  \newpage
  \section*{4. 实验结果}
  \begin{table}[h!]
  	\begin{center}
  		\caption{不同数量级下几种排序算法所耗时间($ms$)}
  		\begin{tabular}{c|c|c|c|c}
  			\toprule
  			\textbf{数组长度} & \textbf{归并排序(并行)} & \textbf{快速排序(并行)} & \textbf{归并排序(非并行)} & \textbf{快速排序(非并行)} \\
  			\midrule
  			$10^4$ & $3$ & $0$ & $0.916$ & $0.838$ \\
  			$10^5$ & $49$ & $5$ & $6$ & $4$ \\
  			$10^6$ & $474$ & $53$ & $84$ & $54$ \\
  			$10^7$ & $5024$ & $563$ & $951$ & $638$ \\
  			$10^8$ & $57867$ & $7130$ & $11069$ & $7276$ \\
  			$10^9$ & $687494$ & $70697$ & $129550$ & $81765$ \\
  			\bottomrule
  		\end{tabular}
  	\end{center}
  \end{table}
  \section*{5. 分析与总结}
  a. 可以看到,两种排序方法使用$openmp$的并行版本并没有达到预期的效率,归并排序的并行版本的效率有大幅的下降,而快速排序则只有一点微小的效率提升。\\
  b. 横向比较发现快速排序的速度在并行实现上依然远高于归并排序。\\
  c. 并行版本的速度没有得到提升的可能原因有三点:首先$openmp$平台为了规划并行消耗了较多的计算资源;其次,每次申请一个新的线程都需要耗费一定的时间;最后,由于$Visual \: Studio$平台上的$openmp$版本较老,不支持$task$功能,因此在递归过程中实际上并没有充分利用计算机上富余的物理内核;如果强行开启$nest$选项,则会由于递归过深创建线程个数过多反而导致效率下降甚至爆栈。\\
  e. 在现代内核上,由于单核的计算速度已经足够快,并且有缓存机制作为保障,使得这种基础算法没有太大的必要使用多线程运行来提升效率,有时候反而会因为多开线程而导致额外的开销。
  \section*{6. 源代码及可执行文件说明}
  a. 源代码存放在$src/psort$文件夹下。\\
  b. 可执行文件存放在$bin/psort$文件夹下。\\
\end{document}
