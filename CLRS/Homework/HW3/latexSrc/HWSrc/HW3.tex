%!TEX program = xelatex
%!TEX TS-program = xelatex
%!TEX encoding = UTF-8 Unicode
\documentclass[12pt]{article}

\usepackage{amsmath}
\usepackage{hyperref}
\usepackage{geometry}
\usepackage{fontspec,xltxtra,xunicode}
\usepackage{setspace}
\usepackage{algorithm}  
\usepackage{algorithmicx}  
\usepackage{algpseudocode}  
  
%\floatname{algorithm}{OPTIMAL-BST}
\renewcommand{\algorithmicrequire}{\textbf{输入:}}
\renewcommand{\algorithmicensure}{\textbf{输出:}}

\defaultfontfeatures{Mapping=tex-text}
\setromanfont{Heiti SC}
\XeTeXlinebreaklocale “zh”

\title{Homework2 Solution}
\author{洪方舟\\Student ID: 2016013259\\Email: \href{mailto:hongfz16@163.com}{hongfz16@163.com}}

\geometry{left=2cm,right=2cm,top=1cm,bottom=2cm}
\renewcommand{\baselinestretch}{1.5}

\begin{document}
  \maketitle
  \section*{Ex8.4-4 \textbf{解:}}
  	使用桶排序的思想,需要将圆分为$n$个相同面积的同心圆环;\\
  	不妨设半径的序列为$a_0,a_1,\ldots,a_{n-1}$,其中$a_0=0$,由面积相等可得,$i\geq1$时
  	\begin{align*}
  	  \pi(a_i^2-a_{i-1}^2) &= \frac{\pi}{n}\\
  	  a_i^2 &= \frac{1}{n}+a_{i-1}^2\\
  	  a_i &= \sqrt{\frac{1}{n}+a_{i-1}^2}
  	\end{align*}
  	下面由数学归纳法证明$a_i=\sqrt{\frac{i}{n}}$\\
  	假设$n=k$时,$a_k=\sqrt{\frac{k}{n}}$成立\\
  	则当$n=k+1$时,$a_{k+1}=\sqrt{\frac{1}{n}+a_k^2}=\sqrt{\frac{k+1}{n}}$\\
  	由数学归纳法可知$a_i=\sqrt{\frac{i}{n}},0\leq i \leq n-1$\\
  	则按照上面描述的半径序列取出同心圆环作为“桶”,则可以达到$\Theta(n)$的排序复杂度。
  \section*{Ex15.5-4 \textbf{解:}}
    \begin{algorithm}  
        \caption{OPTIMAL-BST-IMPROVED}
        \begin{algorithmic}[1]  
            \Require $p,q,n$
            \Ensure $e,root$
            \Function {OPTIMAL-BST-IMPROVED}{$p,q,n$}
                \State let $e[1..n+1,0..n],\omega[1..n+1,0..n],$ and $root[1..n,1..n]$be new tables
                \For{$i=1$ to $n+1$}
                	\State$e[i,i-1]=q_{i-1}$
                	\State$\omega[i,i-1]=q_{i-1}$
                \EndFor
               	\For{$l=1$ to $n$}
               		\For{$i=1$ to $n-l+1$}
               			\State$j=i+l-1$
               			\State$e[i,j]=\infty$
               			\State$\omega[i,j]=\omega[i,j-1]+p_j+q_j$
               			\If{$i==j$}
               				\State $e[i,j]=e[i,j-1]+e[j+1,j]+\omega[i,j]$
               				\State $root[i,j]=j$
               			\Else
               				\For{$r=root[i,j-1]$ to $root[i+1,j]$}
               					\State $t=e[i,r-1]+e[r+1,j]+\omega[i,j]$
               					\If{$t<e[i,j]$}
               						\State $e[i,j]=t$
               						\State $root[i,j]=r$
               					\EndIf
               				\EndFor
               			\EndIf
               		\EndFor
               	\EndFor
               	\State \Return $e$ and $root$
            \EndFunction
        \end{algorithmic}
    \end{algorithm}
    第16行的$for$循环(原书第10行)经过改进后,可以做到第二层循环(第8行)加上第三层循环为$\Theta(n)$的时间复杂度,则可以达到总共$\Theta(n^2)$的复杂度。
\end{document}