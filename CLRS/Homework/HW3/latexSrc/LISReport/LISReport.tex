%!TEX program = xelatex
%!TEX TS-program = xelatex
%!TEX encoding = UTF-8 Unicode
\documentclass[12pt]{article}

\usepackage{amsmath}
\usepackage{hyperref}
\usepackage{geometry}
\usepackage{fontspec,xltxtra,xunicode}
\usepackage{booktabs}
\usepackage{float}
\usepackage{changepage}
\usepackage{setspace}

\defaultfontfeatures{Mapping=tex-text}
\setromanfont{Heiti SC}
\XeTeXlinebreaklocale “zh”

\title{CLRS Ex15.4-6 实验报告}
\author{洪方舟\\Student ID: 2016013259\\Email: \href{mailto:hongfz16@163.com}{hongfz16@163.com}}

\geometry{left=2cm,right=2cm,top=2cm,bottom=3cm}

\renewcommand{\baselinestretch}{1.5}

\begin{document}
  \maketitle
  \section*{1. 实验目的}
  a. 编写程序实现复杂度为$O(nlgn)$的算法寻找数组中最长递增子序列\\
  b. 加深对动态规划思想的认识
  \section*{2. 实验环境}
  操作系统:ubuntu 16.04 LTS\\
  处理器:Intel Core i7-7700k CPU @ 4.20GHz $\times$ 8\\
  编程语言:C++\\
  编译器:g++
  \section*{3. 实验方法}
  a. 考虑长度为$i$的递增子序列末尾数字的最小值\\
  b. 在$dp$数组中第$i$个位置存储长度为$i$的递增子序列末尾数字的最小值,每次更新$dp$数组的时候二分查找不大于待插入数字的最后一个位置\\
  c. 当完成遍历后$dp$数组的长度就是最长递增子序列的长度,通过回退查找可以构造出一个最长递增子序列\\
  d. 使用$O(n^2)$的算法来保证本算法的正确性\\
  e. 使用$Qt$制作方便测试的图形界面
  \section*{4. 实验分析}
  算法中总共只有一个循环,复杂度为$O(n)$,在每一个循环中,需要去$dp$数组中查询插入位置,该插入位置$i$应满足$dp[i]\leq NumberToInsert<dp[i+1]$,使用二分查找的复杂度为$O(lgn)$;综上所述算法总复杂度为$O(nlgn)$;
  \section*{5. 实验总结}
  a. 通过两种算法的对比验证,可知本算法具有一定的正确性\\
  b. 对于$dp$数组的不同的理解方法可以带来不同的算法复杂度
  \section*{6. 源代码及可执行文件说明}
  a. 源代码存放在$/src/LISSrc$\\
  b. 可执行文件存放在$/bin/LIS$\\
  c. 打开可执行文件后是用户界面,可以在输入框内输入一串用空格隔开的数组(若输入不合法将会给出提示),点击$Submit!$按钮后在下方将最长递增序列标红显示,并给出最大长度和运算时间
\end{document}