%!TEX program = xelatex
%!TEX TS-program = xelatex
%!TEX encoding = UTF-8 Unicode

%\documentclass[12pt]{article}
%\usepackage{amsmath}
%\usepackage{hyperref}
%\usepackage{geometry}
%\usepackage{fontspec,xltxtra,xunicode}
%\usepackage{booktabs}
%\usepackage{float}
%\usepackage{changepage}
%\usepackage{setspace}
%\usepackage{listings}

\documentclass{article}
\usepackage{amsmath}
\usepackage{geometry}
\usepackage{CJK}
\usepackage{ctex}
\usepackage{graphicx}
\usepackage{booktabs}
\usepackage{pst-node}
\usepackage{pst-text}
\usepackage{amsmath}
\usepackage{amssymb}
\usepackage{listings}
\usepackage{fancyhdr}
\usepackage[colorlinks]{hyperref}
\usepackage{color}
\usepackage{hyperref}
\usepackage{clrscode}

%\defaultfontfeatures{Mapping=tex-text}
%\setromanfont{Heiti SC}
%\XeTeXlinebreaklocale “zh”

% \definecolor{gray}{rgb}{0.4,0.4,0.4}
% \definecolor{darkblue}{rgb}{0.0,0.0,0.6}
% \definecolor{cyan}{rgb}{0.0,0.6,0.6}
% \definecolor{grey}{rgb}{0.7,0.7,0.7}

\title{String Matching Experiment Report}
\author{洪方舟\\2016013259\\Email: \href{mailto:hongfz16@163.com}{hongfz16@163.com}}
\date{\today}
\geometry{left=2cm,right=2cm,top=2cm,bottom=2cm}

\begin{document}
  \maketitle
  \setlength\parindent{0em}
  \section*{1. 实验目的}
  a. 实现$Brute-Force, KMP, Boyer-Moore$三种字符串匹配算法\\
  b. 要求字母表至少含有26个英文字母,0-9数字以及若干英文标点符号\\
  c. 实现方便测试的图形界面,程序可以输入txt文件做测试\\
  d. 测试比较不同算法在不同的问题规模下的运行效率
  \section*{2. 实验环境}
  操作系统:Windows 10\\
  处理器:Intel Core i7-7700k CPU @ 4.20GHz $\times$ 8\\
  编程语言:C++\\
  IDE: Visual Studio
  \section*{3. 实验方法}
  a. 编写三种算法,在命令行版本中将三者结果相比较来保证算法的正确性\\
  b. 在命令行版本中随机生成不同规模的数据,记录运行时间\\
  c. 使用QT编写方便测试的用户界面\\
  d. 在QT界面中使用可以打开文件的控件来实现打开txt文件的要求
  \newpage
  \section*{4. 实验结果}
  \begin{table}[h!]
  	\begin{center}
  		\caption{不同字符串匹配算法在不同数量级数据下所耗时间($ms$)}
  		\begin{tabular}{c|c|c|c|c}
  			\toprule
  			\textbf{模式串长度} & \textbf{待匹配串长度} & \textbf{$Brute-Force$} & \textbf{$KMP$} & \textbf{$BM$} \\
  			\midrule
  			$10^0$ & $10^6$ & $2$ & $2$ & $9$ \\
  			$10^1$ & $10^6$ & $2$ & $3$ & $2$ \\
  			$10^2$ & $10^6$ & $2$ & $3$ & $0$ \\
  			$10^3$ & $10^6$ & $2$ & $3$ & $0$ \\
  			$10^4$ & $10^6$ & $2$ & $2$ & $1$ \\
  			$10^5$ & $10^6$ & $2$ & $3$ & $2$ \\
  			\midrule
  			$10^0$ & $10^7$ & $19$ & $24$ & $97$ \\
  			$10^1$ & $10^7$ & $19$ & $24$ & $14$ \\
  			$10^2$ & $10^7$ & $19$ & $24$ & $6$ \\
  			$10^3$ & $10^7$ & $19$ & $24$ & $5$ \\
  			$10^4$ & $10^7$ & $19$ & $24$ & $5$ \\
  			$10^5$ & $10^7$ & $18$ & $23$ & $6$ \\
  			$10^6$ & $10^7$ & $16$ & $26$ & $15$ \\
  			\midrule
  			$10^0$ & $10^8$ & $194$ & $239$ & $966$ \\
  			$10^1$ & $10^8$ & $188$ & $233$ & $136$ \\
  			$10^2$ & $10^8$ & $189$ & $253$ & $60$ \\
  			$10^3$ & $10^8$ & $188$ & $232$ & $54$ \\
  			$10^4$ & $10^8$ & $187$ & $232$ & $53$ \\
  			$10^5$ & $10^8$ & $188$ & $232$ & $55$ \\
  			$10^6$ & $10^8$ & $186$ & $235$ & $63$ \\
  			$10^7$ & $10^8$ & $170$ & $266$ & $149$ \\
  			\bottomrule
  		\end{tabular}
  	\end{center}
  \end{table}
  \section*{5. 分析与总结}
  a. 通过上面的运行时间可以发现,暴力方法与KMP算法的运行时间相差不大,总体来说,暴力算法的效率甚至高于KMP算法\\
  b. BM算法在特定的条件下运行效率较高:1. 模式串的长度大于$10^2$;2. 待匹配串与模式串的长度比值大于$100$;\\
  c. 尽管暴力方法的时间复杂度为$O(mn)$,KMP算法的时间复杂度是线性的,但是在随机数据的情况下暴力方法很难达到最坏情况,因此甚至比线性算法运行效率高;另外,由于KMP算法模式串预处理时间复杂度为$O(m)$,因此当模式串规模增长时,KMP算法耗时也相应增长;\\
  d. 由于模式串预处理耗时较多,BM算法在模式串较短时的运行时间较长,无法体现其优势;当模式串长度增长,BM算法的优势也就体现出来,在一定范围内,模式串越长,BM算法效率越高;但是当待匹配串的长度与模式串长度之比接近$10$的时候,模式串预处理的时间劣势就显示了出来\\
  e. 综上,不难发现,由于字符串的随机性,暴力方法的效率非常高,因此的日常使用中完全足够;KMP算法虽然在理论上的时间复杂度优于暴力方法,但是在实际应用中效率很差,因此KMP算法仅仅具有理论意义,而不具备使用价值;BM算法在一定的数据规模范围内效率最高,尤其是长模式串的情况下,应当选用BM算法。
  \section*{6. 源代码及可执行文件说明}
  a. 控制台版本用于不同数据规模自动测试,和手动输入数据测试(如果需要显示结果,请在源代码$main.cpp$文件中注释宏定义$\_NOT\_SHOW\_ANS\_$),使用方法见命令行中提示文字;可执行文件位于$bin/Console$\\
  b. GUI版本用于打开外部$txt$文件进行测试,打开界面后点击$Open$按钮选择载入测试文件,在第二行的文本框中输入模式串,而后提供两种测试模式:$Find First$按钮对应于找到测试文件中第一处出现模式串的位置,$Find All$按钮对应于找到测试文件中所有的模式串位置;在下方的文本框中会显示匹配结果,以及三种方法用时;可执行文件位于$bin/GUI$
\end{document}
